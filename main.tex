\documentclass[conference]{IEEEtran}

\usepackage{cite}
\usepackage{amsmath,amssymb,amsfonts}
\usepackage{algorithmic}
\usepackage{graphicx}
\usepackage{textcomp}
\usepackage{xcolor}

\def\IEEEtitletopspace{18pt}
\def\IEEEtitlebottomspace{6pt}

\begin{document}

\title{Title of Your Paper}

\author{\IEEEauthorblockN{Abdelrahman Alaa, Ahmed Al-Deeb, Ahmed Raafat,
Kareem Abdelnabi, Salah Mohamed}
\IEEEauthorblockA{Department of Systems and Biomedical Engineering, Cairo University}
}

\maketitle

\begin{abstract}
This study aims to investigate how healthcare sytems are managed within
different economic models. We begin by introducing the management of healthcare
within the two main economic systems: capitalism and socialism. After inspecting the
pros and cons of each system and providing examples of how captialist and Socialist
nations manage their healthcare systems, we then present a bried about Islamic Economics
and how an Isalmic economic system can fill the gaps in healthcare systems and the whole
financial system generally left by capitalism and socialism. However, issues regarding applying
Islamic economics will be discuused. Moreover, Examples that managed to apply successful versions
of the Islamic Model will be presented. 
\end{abstract}

\begin{IEEEkeywords}
economics, finance, capitalism, socialism, healthcare, islamic economics,
islamic economic system
\end{IEEEkeywords}

\section{Introduction}

\par 
The main difference between capitalism and socialism 
lies in how each system approaches ownership, control, and distribution of resources. 
Capitalism emphasizes private ownership and a free-market economy, 
where individuals and businesses operate with minimal government interference. 
The focus is on profit generation, and market competition drives innovation and efficiency.
On the other hand, socialism advocates for government control over the economy, 
particularly over essential services like healthcare. 
It seeks to ensure equal distribution of resources and services, 
with the government acting as the central authority to address social and economic inequality.

\par
Capitalism focuses on private ownership and a free market, 
where businesses and people make decisions based on profit. 
The government has a small role. Adam Smith, a famous economist, 
believed that markets should be free, and businesses should compete 
with each other to provide goods and services.

\par
Socialism supports government control of important services, 
including healthcare. Karl Marx, who created the ideas behind socialism, 
argued that capitalism often creates unfairness and wealth gaps. 
He believed that the government should control resources and 
services to make sure everyone gets their fair share.

\section{Providing Healthcare in a Capitalist System}

\par
In capitalism, healthcare is provided by private companies. 
People either pay directly for services or buy private insurance. 
This often means that wealthier people can get better care, 
while poorer people may not have access to quality healthcare.
Private companies compete to provide healthcare. 
This can lead to better services and new medical treatments, 
but it can also result in unequal access to care. 
Wealthy people can afford the best services, 
while poorer people may struggle to get the care they need.

\section{Providing Healthcare in a Socialist System}

\par
In socialism, the government owns and runs healthcare. 
It is paid for through taxes, and everyone is entitled to the same 
healthcare services, no matter how much money they have. 
The goal is to ensure that everyone has access to the care they need, 
no matter their income.However, sometimes the government-run 
system can face inefficiencies or shortages in resources, 
making it harder to provide the best care for everyone.

\subsection{Real World Examples}

\section{Islamic Economics}
Introductory Paragraph
\subsection{How is the Healthcare System Operataed in an Islamic Regime}
\subsection{Filling the Gaps of Capitalism and Socialism}
\subsection{Real World Examples}

\section{Issues within Isalmic Economic Models}
Present the results of your research. Use tables, figures, and graphs to support your findings.

\section{Conclusion}
Summarize the main findings of your research and suggest future work.

\section*{Acknowledgment}
Acknowledge any funding sources, collaborators, or other contributions to your research.

\bibliographystyle{IEEEtran}
\bibliography{references}

\end{document}