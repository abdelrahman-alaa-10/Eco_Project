\par
The Islamic model offers a balanced approach to healthcare, effectively addressing the shortcomings of both capitalism and socialism. Unlike capitalism, where healthcare is often treated as a commodity and driven by profit, the Islamic model ensures equitable access by eliminating exploitative practices such as excessive pricing and interest-based financing. Mechanisms like Zakah (obligatory almsgiving) and Waqf (charitable endowments) provide sustainable funding for healthcare, ensuring affordability and inclusivity, especially for underprivileged populations. At the same time, the Islamic model avoids the inefficiencies often associated with socialism, such as resource misallocation and lack of innovation stemming from centralized state control. By leveraging decentralized financing tools like Sukuk (Islamic bonds) and profit-sharing partnerships such as Mudarabah and Musharakah, the model encourages private sector participation while maintaining a strong emphasis on societal welfare.  