\par
While the healthcare system during the Islamic Golden Age was groundbreaking and set the foundation for modern medicine, it faced certain limitations due to historical, technological, and social contexts rather than inherent flaws.
Hospitals (bimaristans) were primarily located in major urban centers such as Baghdad, Cairo, and Damascus, leaving rural and remote areas underserved. Efforts to bridge this gap, such as traveling physicians and mobile clinics, were commendable but insufficient for widespread coverage.

\par
The healthcare system depended heavily on waqf (charitable endowments), which were subject to the wealth and generosity of donors. Economic downturns or mismanagement of these funds could disrupt healthcare services. Instances of inefficiency or misuse in the allocation of waqf resources further strained hospital operations.
Medical advancements were limited by the technology of the time. Treatments for certain diseases and advanced procedures, such as surgeries, were often rudimentary. Cultural and religious norms sometimes restricted practices like anatomical dissections, which impeded deeper understanding of human anatomy.

\par
Although mental healthcare was available in some hospitals, societal stigma surrounding mental illness often prevented individuals from seeking treatment. Moreover, mental health practices were still in their infancy compared to physical health treatments.
While Islamic principles emphasized care for all, cultural barriers occasionally limited women’s access to healthcare, especially for conditions requiring male physicians. Female physicians existed but were less common, reducing the availability of gender-sensitive care.

\par
Urban hospitals often struggled with overcrowding and shortages of medical supplies. Centralized reliance on waqf management could delay decision-making, affecting the efficient allocation of resources.
Medical education and practices varied across regions due to the absence of standardized curricula and protocols. Knowledge dissemination was slow, relying on manuscripts and personal mentorship, which hindered consistency in the quality of care.

\par
Advanced treatments and specialized care were largely confined to major urban hospitals, leaving smaller towns and villages reliant on general practitioners with limited resources.
Although hospitals prioritized hygiene and individual care, broader public health measures—such as sewage systems and clean water supplies—lagged behind. These gaps in infrastructure hindered the prevention and control of diseases.
While hospitals were open to all, occasional biases based on social class or connections may have influenced treatment quality. Documentation does not suggest widespread discrimination but points to localized disparities.