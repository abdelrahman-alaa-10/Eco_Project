\par
High out-of-pocket expenses often prevent low-income individuals from accessing essential healthcare services, leading to significant disparities in health outcomes across socioeconomic groups. Many individuals remain uninsured or underinsured, further exacerbating health inequities. In countries like the United States, the absence of universal healthcare coverage leaves millions unable to afford basic medical care. The capitalist model’s emphasis on profit frequently drives up healthcare costs without corresponding improvements in outcomes. For example, the U.S. spends nearly twice as much on healthcare per capita as other OECD countries, yet it lags behind in key health metrics. 

\par
The complexity of navigating multiple private insurance plans contributes to high administrative costs, diverting resources away from patient care. Patients often lack the information needed to make informed healthcare decisions, which can lead to suboptimal outcomes. This knowledge gap is sometimes exploited by providers who may recommend unnecessary treatments. Moreover, the system's focus on high-profit treatments rather than preventive care results in poorer long-term health outcomes. Essential public health initiatives, which are less profitable, frequently receive insufficient funding. Significant disparities in care quality also persist, with wealthier regions benefiting from better-funded facilities and staff, while poorer areas struggle with underfunded healthcare systems and shortages of skilled professionals. The increasing financialization of healthcare prioritizes short-term profits over sustainable health outcomes, undermining the overall quality of care.

\par
In extreme cases, patients turn to crowdfunding to cover medical expenses, underscoring systemic failures in public health funding and insurance coverage. Efforts to reform the system and improve access are often resisted by powerful industry stakeholders, who prioritize profit over equity and efficiency. During economic downturns, governments may reduce healthcare spending, further widening inequalities and limiting access to essential services for vulnerable populations.