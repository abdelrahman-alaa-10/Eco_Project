\par 
The primary difference between capitalism and socialism lies in their approaches to ownership, control, and the distribution of resources. Capitalism emphasizes private ownership and a free-market economy, where individuals and businesses operate with minimal government interference. The focus is on profit generation, with market competition driving innovation and efficiency. In contrast, socialism advocates for government control over the economy, particularly essential services like healthcare. It aims to ensure an equitable distribution of resources and services, with the government acting as a central authority to address social and economic inequalities.

\par
Capitalism champions private ownership and free markets, allowing businesses and individuals to make decisions based on profit motives. The government's role is limited. Adam Smith, a prominent economist, supported the idea of free markets, arguing that competition among businesses would lead to better goods and services. On the other hand, socialism supports government control over key services, including healthcare. Karl Marx, the foundational thinker behind socialism, criticized capitalism for creating inequality and wealth disparities. He advocated for government control of resources and services to ensure a fair distribution among all members of society.