\par 
The main difference between capitalism and socialism 
lies in how each system approaches ownership, control, and distribution of resources. 
Capitalism emphasizes private ownership and a free-market economy, 
where individuals and businesses operate with minimal government interference. 
The focus is on profit generation, and market competition drives innovation and efficiency.
On the other hand, socialism advocates for government control over the economy, 
particularly over essential services like healthcare. 
It seeks to ensure equal distribution of resources and services, 
with the government acting as the central authority to address social and economic inequality.

\par
Capitalism focuses on private ownership and a free market, 
where businesses and people make decisions based on profit. 
The government has a small role. Adam Smith, a famous economist, 
believed that markets should be free, and businesses should compete 
with each other to provide goods and services.

\par
Socialism supports government control of important services, 
including healthcare. Karl Marx, who created the ideas behind socialism, 
argued that capitalism often creates unfairness and wealth gaps. 
He believed that the government should control resources and 
services to make sure everyone gets their fair share.