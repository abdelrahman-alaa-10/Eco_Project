\par
This study has explored the management of healthcare under different economic systems, highlighting the strengths and weaknesses of capitalism, socialism, and the Islamic economic model. Capitalism, with its emphasis on private ownership and market competition, fosters innovation but often leads to inequality in healthcare access, prioritizing profit over universal care. Socialism, on the other hand, ensures equitable access to healthcare through government control but faces challenges of inefficiency, resource misallocation, and limited innovation.
The Islamic economic model offers a unique middle ground, drawing from the strengths of both systems while addressing their weaknesses. By eliminating exploitative practices such as interest-based financing and implementing mechanisms like Zakah and Waqf, it ensures affordability and inclusivity in healthcare. Furthermore, its decentralized and community-driven approach promotes sustainable resource allocation and encourages both private and public sector collaboration.

\par
The analysis of real-world examples has demonstrated how principles of the Islamic model, when implemented effectively, can create a balanced healthcare system that upholds both equity and efficiency. However, challenges such as integrating these principles into modern economies and addressing global disparities remain critical areas for further exploration.
In conclusion, while no economic model is without flaws, the Islamic model provides a compelling framework for addressing the gaps in healthcare systems under capitalism and socialism. Its focus on ethical financial practices, societal welfare, and inclusivity underscores its potential as a viable alternative for ensuring equitable healthcare access in a rapidly changing world.